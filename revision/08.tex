\documentclass[12pt,a4paper]{caspset}
% set 1-inch margins in the document
\usepackage[left=3.18cm,right=3.18cm,top=2.54cm,bottom=2.54cm]{geometry}
\usepackage{lastpage}
\usepackage{graphicx}
\usepackage{amsmath,amsfonts,amsthm,amssymb}
\usepackage{setspace}
\usepackage{fancyhdr}
\usepackage{lastpage}
\usepackage{extramarks}
\usepackage{chngpage}
\usepackage{soul}
\usepackage[usenames,dvipsnames]{color}
\usepackage{graphicx,float,wrapfig}
\usepackage{ifthen}
\usepackage{listings}
\usepackage{courier}
\usepackage{multimedia}
\usepackage[toc,page,title,titletoc,header]{appendix}
\usepackage{color, soul}
\usepackage{wrapfig}
%\usepackage{Picinpar}
\usepackage{xypic}
\setulcolor{red}

\newcommand{\hmwkTitle}{English writing}
\newcommand{\hmwkSubTitle}{My Hometown} % No subtitle, so this will be excluded
\newcommand{\hmwkDueDate}{\today}
\newcommand{\hmwkClass}{College of English}
\newcommand{\hmwkClassTime}{Thu.{~}08:00}
\newcommand{\hmwkClassInstructor}{Philip Dykshoorn}
\newcommand{\hmwkAuthorName}{Zhou Lvwen}


%% Setup the header and footer
\pagestyle{fancy}                                                       %
\lhead{\hmwkAuthorName}                                                 %
\chead{\hmwkClassInstructor\ ~\ \hmwkClassTime: \hmwkTitle}  %
\rhead{Page\ \thepage\ of\ \protect\pageref{LastPage}}            %
%%%%%%%%%%%%%%%%%%%%%%%%%%%%%%%%%%%%%%%%%%%%%%%%%%%%%%%%%%%%%
\newlength\picwidth
\setlength\picwidth{0.23\textwidth}
\newlength\pichigh
\setlength\pichigh{1.277\picwidth}

% info for header block in upper right hand corner
\name{Zhou Lvwen{~}201128000718065}
\class{Class B-165}
\assignment{Assignment \# 08}
\duedate{May 2, 2013}
\setlength{\parskip}{1.2em}
\setlength\parindent{0pt}
\linespread{1.3}
\begin{document}

\problemlist{\large One child policy: not a good permanent solution}

China, as the most populous country in the world for a long time, faced pressures of overpopulation. The Chinese government established one child policy to  alleviate social, economic, and environmental problems in China. Until now, the policy has been estimated to have reduced the population in the country of 1.5 billion by as much as 400 million people. Although the policy seams really worked in reducing the population, I don't think it is a good permanent solution. 

Firstly, it is not good for children's health that growing up without siblings. Children are becoming more spoiled than other family configurations. Parents spoil their children because they only have one.  The final result is that the only children aren’t known for handling adversity especially well. However, they must handle much adversity when they graduate.

Secondly, the policy cause aging population  in China. As the aging society is coming, the pension pressure is pop out increasingly. Many newly married couples, as both only child, often shouldering the task of taking care of four older parents without siblings to share the responsibility. It is a heavy burden for young people and may have negative effects on personal development. 

Lastly, the policy has caused the gender imbalance of six males for every five females among babies form birth through children four yeas of age. Because of the limitation of one child, many families, especially in the countryside, have chosen to abort foetuses, preferring to have boys. Abortion, neglect, abandonment and even infanticide have been known to occur to female infants.

In conclusion, one child policy is not a good permanent solution to reducing population, because of its serious side effects, such as harmful to children, a rapidly aging population and serious gender imbalance.

\end{document}
