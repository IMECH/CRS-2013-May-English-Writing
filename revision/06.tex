\documentclass[12pt,a4paper]{caspset}
% set 1-inch margins in the document
\usepackage[left=3.18cm,right=3.18cm,top=2.54cm,bottom=2.54cm]{geometry}
\usepackage{lastpage}
\usepackage{graphicx}
\usepackage{amsmath,amsfonts,amsthm,amssymb}
\usepackage{setspace}
\usepackage{fancyhdr}
\usepackage{lastpage}
\usepackage{extramarks}
\usepackage{chngpage}
\usepackage{soul}
\usepackage[usenames,dvipsnames]{color}
\usepackage{graphicx,float,wrapfig}
\usepackage{ifthen}
\usepackage{listings}
\usepackage{courier}
\usepackage{multimedia}
\usepackage[toc,page,title,titletoc,header]{appendix}
\usepackage{color, soul}
\usepackage{wrapfig}
%\usepackage{Picinpar}
\usepackage{xypic}
\setulcolor{red}

\newcommand{\hmwkTitle}{English writing}
\newcommand{\hmwkSubTitle}{My Hometown} % No subtitle, so this will be excluded
\newcommand{\hmwkDueDate}{\today}
\newcommand{\hmwkClass}{College of English}
\newcommand{\hmwkClassTime}{Thu.{~}08:00}
\newcommand{\hmwkClassInstructor}{Philip Dykshoorn}
\newcommand{\hmwkAuthorName}{Zhou Lvwen}


%% Setup the header and footer
\pagestyle{fancy}                                                       %
\lhead{\hmwkAuthorName}                                                 %
\chead{\hmwkClassInstructor\ ~\ \hmwkClassTime: \hmwkTitle}  %
\rhead{Page\ \thepage\ of\ \protect\pageref{LastPage}}            %
%%%%%%%%%%%%%%%%%%%%%%%%%%%%%%%%%%%%%%%%%%%%%%%%%%%%%%%%%%%%%
\newlength\picwidth
\setlength\picwidth{0.23\textwidth}
\newlength\pichigh
\setlength\pichigh{1.277\picwidth}

% info for header block in upper right hand corner
\name{Zhou Lvwen{~}201128000718065}
\class{Class B-165}
\assignment{Assignment \# 06}
\duedate{April 11, 2013}
\setlength{\parskip}{1.2em}
\setlength\parindent{0pt}
\linespread{1.3}
\begin{document}

\problemlist{\large The Lost Son}

My father was born in rural north of Jiangsu province in 1955. He has two elder brothers, one younger brother  and one younger sister.  When my father was three years old, the famines haunted China and lasted three years from 1958 to 1961. The Chinese refer to this era as "the three years of natural disaster." Though the famines was over in 1961, the effects of the natural disaster endure several years. 

In 1965, The family could not support so many children. Towards the end of the year, my father and his two elder brothers had to left home and go to beg for money or food in south Jiangsu for living. South Jiangsu is relatively rich, and Zhenjiang is one of cities in Jiangsu. One day in 1966, my father got separated from his two brothers when begging on the street in Zhenjiang. My father was only 11 years old that year. Fortunately, after drifted helplessly for a few days, my father was adopted by a family. 

In 1968, two years later after my father lost. My grandpa set off in search of my father.  My grandpa came to Zhenjiang, and searched in all quarters but still couldn't find his son. One day, my father was washing vegetables near the river, and saw a person on the bridge. My father recognized immediately that was his father, and shouted: “papa, papa”.  My father was finally back to home with his father and reunited with his family.
\end{document}
